\documentclass{article}
\usepackage{graphicx}
\usepackage{hyperref}
\usepackage[a4paper, margin=1in]{geometry}
\usepackage{breakcites}
\usepackage{subcaption}
\usepackage{multicol}
\usepackage{booktabs}
\usepackage{longtable}
\usepackage{lineno}
\usepackage{authblk}

\linenumbers

\begin{document}

\title{Mapping Africa's Ecological Safety Nets: Where Should Conservation Efforts Be Targeted to Sustain Ecosystem Services for Nutritional Resilience to Climate Change?}

\author[1,2,*]{Cooper, Matthew}
\author[2]{Silva, Julie}
\author[3]{Sahyoun, Nadine}
\author[4]{Zvoleff, Alex}
\author[2]{Hansen, Matthew}
\author[2]{Brown, Molly}

\affil[1]{T.H. Chan School of Public Health, Harvard University}
\affil[2]{Department of Geographical Sciences, University of Maryland College Park}
\affil[3]{Department of Nutrition and Food Science, University of Maryland College Park}
\affil[4]{Betty and Gordon Moore Center for Science, Conservation International}
\affil[*]{Corresponding Author: mcooper@hsph.harvard.edu}

\maketitle
\begin{abstract}

In Africa, where millions of households depend on rainfed agriculture to produce food for their own consumption, climate change is a major threat to food security.  A large literature suggests that ecosystem services can be an asset in the face of climate change by shielding cropland from the effects of droughts and heat waves, while also providing wild foods when yields are low.  However, much of the work focusing on the safety net provided by uncultivated land has been conducted in highly localized and site-specific case studies which often rely on hypothetical or retrospective analyses.  To date, there has been little empirical and spatially explicit work on which areas provide the most benefit to local food security.  In this study, we combine data on nutrition outcomes from 221,225 children in agrarian communities across 32 African countries with historical observations of land cover and climate shocks to test the hypothesis that uncultivated land can act as a safety net in certain contexts.  We find that in woodland, semi-humid agro-ecological zones in Africa, children in areas with more uncultivated land cover are less drought affected than those in areas with more agricultural land cover.  Finally, we map where conservation interventions could have the largest impact on improving nutritional resilience to future droughts, and compare our results to priority areas for conserving biodiversity to identify African landscapes where conservation could provide multiple benefits.

\end{abstract}

\section{Introduction}

Currently, an estimated 58.8 million African children, representing nearly one third of the continent's under-5 population, suffer from chronic undernutrition \cite{unicef2019}.  While progress has been made in the past several decades to improve nutrition and food security outcomes, climate change threatens to stall or even reverse current trends \cite{FAO2018}.  As climate change continues, the frequency and intensity of meteorological extremes will affect food production, ultimately harming food security and nutrition for many vulnerable communities \cite{niles2020climate}.  Africa is particularly vulnerable to these changes, as an estimated 95\% of agriculture is rainfed \cite{Wani2009} and about 65\% of households produce food for their own consumption \cite{Runge2004}.

One factor that can play a major role in fostering food systems that are resilient to climate shocks is the presence of ecosystem services provided by uncultivated areas \cite{Reed2016, Pascual2017, Daily2008}.  These areas provide a suite of regulating services that can buffer agricultural yields from the effects of shocks.  For example, natural vegetation can provide shade and cooler temperatures during heat waves, absorb water and protect against erosion during floods, as well as retain soil moisture during droughts \cite{Siriri2013, Lott2009}.  Furthermore, uncultivated areas can provide habitat for pollinators and species that regulate pest outbreaks \cite{Karp2013}.  Beyond regulating services, uncultivated land provides provisioning services in the form of wild foods and other inedible products that can support local incomes and food security when agricultural output is low \cite{friant2019life, morgan2020secret, powell2015improving, Assogbadjo2012}.

A great deal of literature has focused on the benefit that ecosystem services can provide, although much of this work has relied on studies that are site specific.  For example, detailed work conducted in case studies across Africa have found instances of ecosystem services improving nutrition \cite{Golden2011}, regulating crop pests \cite{Girma2000}, improving yields through pollination \cite{Gemmill-Herren2008, Munyuli2012}, and improving soil nutrient quality \cite{Sileshi2012, Boffa2000, Siriri2009}.  Some work that is particularly relevant to climate resilience has found that natural land cover can improve soil water storage \cite{Siriri2013, Lott2009}, but nevertheless few empirical studies have observed how ecosystem services affect human outcomes \textit{in situ} during climate shocks.  Rather, most studies that focus on ecosystems as a form of climate resilience use surveys that ask respondents if they would rely on ecosystem services in the event of a hypothetical shock \cite{Robledo2012}, with some studies indicating that many people do not think of ecosystem services as a safety net that they would rely on during shocks \cite{Wunder2014}.

Building on all these case studies, a growing body of work has drawn on Demographic and Health Surveys from across Africa and the developing world to assess whether the benefits provided by various ecosystem services can be observed at scale.  This work has shown that forest cover is associated with improved dietary diversity \cite{Ickowitz2014, Rasolofoson2018}, that forested watersheds are associated with less diarrheal disease \cite{Herrera2017}, and that protected areas are associated with a number of health and economic benefits \cite{naidoo2019evaluating}.  However, while these studies have found large-scale associations between environmental variables and positive human outcomes, little work has examined spatial heterogeneities in these associations in order to examine climate resilience or inform conservation priority setting.

While a large body of research attests to the fact that ecosystem services play an important role in food production and nutrition, especially for smallholder farmers, comparatively little work in the field of environmental conservation has been conducted to identify areas where conservation interventions could lead to improved food security and nutrition outcomes.  This is in spite of the fact that the practice of conservation relies heavily on mapping for priority setting - for example, mapping ecosystem services such as carbon sequestration and storage \cite{Kim2016} or water provision \cite{immerzeel2020importance} as well as mapping biodiversity hot spots \cite{holland2012conservation}.  Thus, conducting environmental epidemiology using large, geolocated datasets on human well-being like the DHS could be useful for mapping which landscapes areas contribute the most to human well-being and further catalyze conservation investment, as well as identify locations where conservation interventions could lead to synergies between Sustainable Development Goals (SDGs) related to environmental conservation (13 \& 15) and human well-being (1 \& 2), two goals that are often perceived to be in conflict \cite{moore2016improving, mcshane2011hard}.

This paper aims to fill that research gap by examining the benefit that uncultivated land cover provides specifically to nutrition outcomes during droughts.  This study goes beyond testing for broad associations, but also examines how the relationship between climate shocks, uncultivated land cover, and rainfall varies across agro-ecological zones (AEZs) to identify areas where uncultivated land cover provides the greatest benefit to child nutrition outcomes and inform conservation priority setting.

\section{Theoretical Framework}

\subsection{Land Cover and Ecosystem Services}
The ecosystem services provided by nature are highly varied and operate across different spatial scales.  They are typically classified into provisioning, supporting, regulating, and cultural services \cite{Martinez-Harms2012}, although other typologies exist \cite{Fisher2008}.  A common approach for mapping ecosystem services is to focus on land cover types, especially when primary data is unavailable \cite{Martinez-Harms2012}.  One approach is to analyze each land cover type as providing a ``bundle" of associated ecosystem services \cite{Raudsepp-Hearne2010}.  Thus, in an African context, cultivated land provides primarily food crops as a service, as well as grazing in the off-season, and inedible crop residue for building materials; grasslands provide grazing for livestock as well as habitat for pollinators and pest regulation services; and forests provide a variety of wild foods, soil formation, water quality regulation, and non-timber forest products.  This framework is especially useful for analyzing trade-offs: as natural vegetation is cleared to make room for crop production, the increase in food crops necessitates a decrease in habitat for pollinators and wild food species, as well as the regulating services provided by uncultivated land.  Conversely, as agricultural land is abandoned, it stops providing food crops but can become available again for services such as wild food provision, water quality regulation and erosion protection, although the types and abundance of ecosystem services provided vary significantly depending on vegetation succession and management regimes \cite{wessels2019mapping}.  Supporting this framework that uses land cover as proxy for ecosystem services, previous work has shown that uncultivated land is one of the best geographic predictors of whether households in Africa report collecting both wild foods as well as other provisioning ecosystem services \cite{Cooper2018a}.

\subsection{Uncultivated Land and Commons}
The regulating and supporting services provided by uncultivated land, such as soil formation, pollination, and water retention are, by their very nature, beneficial across boundaries of property and ownership.  However, in cases when land is privately held, provisioning services such as food crops or timber only provide benefits to landowners, who reserve the right to collect these goods.  

In Africa, uncultivated land is often held as a commons, providing resources to multiple members of a community rather than just one landowning household, although specific practices of land tenure, ownership, access rights, and communal domain vary widely across cultural contexts \cite{Wily2008}.  This means that not only regulating and supporting services but even provisioning services such as wild foods and fuelwood provided by uncultivated land are available to many members of a community.  Thus, these areas are especially critical for the poorest members of communities, and these commons are often framed as ``possibly the only capital asset of the poor" \cite{Wily2008}.  Furthermore, empirical research has shown that provisioning services provided by such areas are critical for the livelihoods of women, migrants, and other marginalized groups in rural Africa \cite{Coulibaly-Lingani2009, Pouliot2013}.

Thus, as cropland expands into previously uncultivated areas in Africa due to pressures of both population growth and agricultural commodification \cite{Rudel2013, Laurance2014}, commons and the services they provide for communities and the poor are becoming increasingly depleted.  The conversion of communal land to privately held, cultivated land often happens with no benefit to marginalized community members because communally held land and commons are not well-recognized or protected by African legal systems \cite{Wily2011}.  Similarly, as agricultural land is abandoned and is reforested, provisioning ecosystem services can become publicly available to communities again, especially when the land is managed in ways that maximize ecosystem services \cite{Laris2008, Eldridge2011, Venter2018}.  Conservation interventions that engage local communities, such as community based forest management, provide a framework to prevent the loss of commons that are an important resource for the poorer members of rural African communities \cite{bray2003mexico}.

\section{Data Sources}

\subsection{Nutrition Data}
For this analysis, we use data from Demographic and Health Surveys (DHS) from throughout Africa.  The DHS is often considered the ``gold standard'' of data on health and nutrition from developing countries and is often used in environmental health studies, because the GPS coordinates associated with each DHS site combined with the date of the survey make it possible to infer the environmental context at the time and location of the survey \cite{Brown2014, enenkel2020predict}.  We utilize all surveys from sub-Saharan Africa that met the following criteria at the time of the study: (1) they have geolocated coordinates, to facilitate the extraction of climate conditions and local land cover at the site of each DHS site, (2) they have data on child nutrition outcomes, and (3) they have data on relevant household and individual co-variates of malnutrition.

As our metric of child nutrition, we use Height-for-Age Z-scores (HAZ scores).  This is an indicator of stunting, a consequence of long-term malnutrition, and has been collected in the majority of DHS surveys for decades.  HAZ scores are derived by comparing the height of a child under five years of age to the distribution of heights of well-nourished children of the same age and gender.  While natural variation in human height makes it impossible to diagnose any one individual as stunted \cite{Perumal2018}, stunting can be defined at the population level as the percentage of a population with an HAZ score less than -2.  While human populations do vary in potential attainable height, for children under 5, differences in height are mostly explained by environmental and dietary conditions \cite{Habicht1974}.

\subsection{Drought Data}
For our data on drought, we use precipitation data from the Climate Hazards Infrared Precipitation with Stations (CHIRPS) dataset \cite{Funk2015} and temperature data from Princeton University derived from a land surface re-analysis model \cite{Sheffield2006}.  Because direct observations of long-term climate conditions in Africa are scarce, both of these datasets rely on remote sensing in combination with ground observations as well as land surface modeling to infer meteorological conditions across space.

Using monthly estimates of precipitation as well as average daily monthly maximum and minimum temperatures, we calculate the monthly water balance using the Hargreaves method \cite{Hargreaves1982} and then derive the 24-month Standardized Precipitation-Evapotranspiration Index (SPEI) \cite{Begueria2014}.  This metric compares the water balance over the previous 24 months and compares it to long-term trends in that location, deriving an index that can be interpreted like a Z-Score.  In previous studies of precipitation anomalies and child malnutrition, the SPEI calculated for the 24 months before a survey was the best predictor of child stunting \cite{Cooper2019a}.  Because the SPEI accounts for both precipitation anomalies as well as water lost through heat-induced evapotranspiration, it can characterize meteorological and hydrological droughts, both of which are expected to become more common under climate change \cite{Dai2013}.

While drought has a strong and clear impact on children's nutrition status in many parts of Africa, excessive rainfall can also affect stunting \cite{Cooper2019a, dimitrova2020monsoon}.  To focus only on the effects of drought relative to normal periods, we exclude from our analysis children observed during relatively high levels of rainfall (SPEI \textgreater 1).

\subsection{Land Cover}
For data on land cover near a DHS site, we use a dataset created by the European Space Agency Climate Change Initiative \cite{Defourny2017}, which is available annually for the years 1992 to 2015 at a 300m resolution for 22 distinct land cover classes.  For children observed outside the period of 1992 to 2015 (3\% of children), data from the closest available year was used. For uncultivated land providing regulating, supporting, and communal provisioning ecosystem services, we use all forms of tree, shrub and herbaceous cover, as well as shrubland, grassland, and water bodies.  Additionally, for mosaic land cover types with both cropland and natural vegetation, we counted each pixel as cultivated if it contained more than 50\% cropland and uncultivated if it contained less than 50\% cropland.  Finally, we do not count urban, bare, or permanent snow and ice areas as uncultivated land, as they do not provide most of the local ecosystem services that uncultivated land cover types do.

As our metric for the availability of ecosystem services, we determine the fraction of land within 15 km of each DHS site that was uncultivated at the time of the survey.  We use a 15 km radius for three reasons.  For one, DHS sites are spatially distorted to preserve respondent anonymity, with 99\% of sites displaced by up to 5 km and 1\% of sites displaces by up to 10 km \cite{Grace2012}.  Thus, a 15 km radius more accurately captures landscape-scale land cover characteristics, because the land cover in the immediate vicinity of a community can't be known.  We also focus on a 15 km, landscape-scale area because many ecosystem services flow over large scales, especially abiotic resources that move through space, such as water, as well as ecosystem services from animals, such as bushmeat and pollination \cite{Lopez-Hoffman2010}.  Finally, many livelihood strategies require traveling significant distances to farm, graze livestock or to collect resources, especially as when resources are scarce \cite{Felardo2016, Arku2010}.

Having derived nearby land-cover categories for each DHS cluster, we exclude sites from our analysis that have greater than 1\% of nearby land cover as urban (19.1\% of the original data) or greater than 5\% of nearby land cover as water (14.1\% of the original data).  This is to ensure that we are basing our analysis only on rural, agrarian households that are largely dependent on rainfed agriculture and ecosystem services from non-agricultural areas, rather than households that have livelihoods based on off-farm labor (such as those in urban areas) or livelihoods based on fishing (such as those near coasts or large bodies of water).  Excluding DHS clusters that were either observed during a significantly wet period (SPEI \textgreater 1) or in urban or coastal areas, yields a dataset of 221,885 observations, or 59.6\% of the original 372,197 observations.

\subsection{Agro-Ecological Zones}
Because farm systems, ecosystem services, and the nutritional response to shocks vary according to local biophysical factors, especially temperature, precipitation and elevation, we analyze the effect of ecosystem services in providing drought resilience at the scale of agro-ecological zones (AEZs) \cite{dimitrova2020monsoon}.  We use AEZs rather than other potential groupings, such as livelihood zones, because the response of agriculture to drought and the ecosystem services that uncultivated areas can provide are primarily determined by biophysical conditions.  Furthermore, most data on livelihood zones available at a continental scale is broadly similar to any AEZ characterization \cite{Lynam2002}.  Using the FAO methodology \cite{Fischer2006} AEZs are defined by elevation and length of growing period, where the growing period is defined as days where precipitation plus moisture stored in the soil exceeds half of potential evapotranspiration \cite{Fischer2006}.  In cases where there are ample observations (Savanna and Woodland), we disaggregate each zone into roughly contiguous northern and southern hemisphere zones.  Conversely, in the case of arid zones where there were fewer observations, we aggregated across across the entire continent to create one discontinuous zone, assuming that the relationships between drought, ecosystem services, and nutrition outcomes are comparable across all of arid Africa.  For clarity and simplicity, we label each AEZs by its associated biome or vegetation community, rather than the AEZ \textit{per se} (i.e., we use ``Woodland" instead of ``Semi-Humid Warm Tropical," even though the latter is the nomenclature used by the FAO).  In the end, each zone in our analysis had over 10,000 child nutrition observations from multiple countries and surveys (See Table \ref{table:AEZtab}).

\begin{table}[h]
	\begin{center}
	\begin{tabular}{l | r | r | r}
		AEZ & Children & Countries & Surveys \\
		\hline
		Arid & 11,739 & 9 & 25\\
		Tropical Forest & 20,203 & 17 & 38 \\
    Montane & 56,504 & 18 & 45 \\
		Northern Savanna & 58,392 & 14 & 41 \\
		Northern Woodland & 32,815 & 15 & 42 \\
		Southern Savanna & 19,465 & 9 & 21 \\
		Southern Woodland & 22,767 & 11 & 23 \\
	\end{tabular}
\caption{Number of child nutrition observations per AEZ}
\label{table:AEZtab}
\end{center}
\end{table}

\begin{figure}[h]
	\centering
	\includegraphics[width=0.8\linewidth]{AEZ_Sites.png}
	\caption{Agro-ecological zones and DHS sites included in the study.}
	\label{fig:AEZmap}
\end{figure}

\section{Methods}
For this analysis we model how access to ecosystem services affects the vulnerability of nutrition to drought in each agro-ecological zone.  We use a special class of Generalized Additive Model (GAM) known as a Varying-Coefficient model \cite{Wood2017} with a smooth spline to model how the impact of droughts on HAZ scores varies according the amount of nearby uncultivated land cover.  Furthermore, we use Covariate Balancing Generalized Propensity Scoring (CBGPS) \cite{imai2014covariate} to control for the effects of other geographic factors that affect drought vulnerability and may be correlated with land cover and land use, including population density, subnational GDP per capita, access to larger cities, international trade.

\subsection{Covariate Balancing Generalized Propensity Scoring}
A number of factors are associated with the presence or absence of uncultivated land cover that also affect drought vulnerability.  Thus, to be able to infer that it is uncultivated areas and the ecosystem services they provide that are having a causal effect on reducing drought vulnerability, it is important to control for these variables.  Propensity score weighting is a popular method to deal with this issue; however, most traditional methods involve a binary treatment variable, which must be dichotomized if it is initially measured in continuous terms \cite{Hirano2003, Robins2000}.  Because our treatment variable, uncultivated land, is continuous, and we have no theoretical priors on how it could be dichotomized, we opt instead to use Covariate Balancing Generalized Propensity Scoring (CBGPS), which can be used for continuous treatments and is more robust to mis-specification \cite{Fong2018}.  Moreover, we use the non-parametric method to estimate the generalized propensity score, which finds weights that leave each confounding variable uncorrelated with the treatment variable, while maximizing the empirical likelihood of observing the data.  The non-parametric approach makes it possible to avoid assumptions about the functional form of the propensity score, but is more computationally costly \cite{Fong2018}.

We balance for demographic and economic factors that can influence both drought vulnerability as well as land cover.  These are: 1) population, from the WorldPop project \cite{Tatem2017}, which can affect land cover by increasing pressure for agricultural production \cite{ouedraogo2010land}, as well as drought vulnerability by increasing access to off-farm labor opportunities but also increasing pressure for resources; 2) subnational GDP per capita \cite{Kummu2018}, which can drive agricultural expansion and deforestation, especially in developing countries \cite{culas2012redd}, while also decreasing drought vulnerability \cite{Carrao2016}; 3) national imports per capita \cite{WorldBank2017}, which can drive agricultural expansion \cite{Meyfroidt2013} while also increasing food access when local food production is low \cite{janssens2020global}; and 4) time to travel distance to major cities \cite{Weiss2018, Uchida2008}, which is an indicator of roads and markets, which can both foster deforestation and agricultural expansion \cite{barber2014roads} as well as buffer child nutrition from the effects of droughts \cite{Shively2017}.

After using the non-parametric CBGPS methodology to generate weights for each of these variables with respect to the availability of uncultivated land, we tested to see whether the correlation between these variables and uncultivated land cover decreased \cite{Fong2018}.  We run the algorithm separately for each AEZ in our analysis.  To conduct the balancing we use the CBPS package for R \cite{Fong2018a}, with the default value of $0.1/N$ for the tuning parameter $\rho$, which moderates the trade-off between completely reducing correlation and avoiding extreme outlier weights.  Finally, as a robustness check, we assessed whether censoring extreme weights at the 80th and 90th percentile would affect our model estimates.

\subsection{Modeling Framework}
Having derived weights for the propensity of each observation to have uncultivated land in its vicinity, we then model nutrition outcomes as a function of the local 24-month SPEI score, where the coefficient for SPEI is modeled as a function of uncultivated land cover, controlling for typical household and individual factors as well as the spatially-varying baseline rate of malnutrition using a spherical spline to control for spatial autocorrelation.  This is a specific form of Generalized Additive Model \cite{Hastie1986} known as a varying coefficient model \cite{Wood2017}.  Specifically, or model takes the following form:

\begin{equation}
  y_{ija} = \beta_0 + \beta X_{ija} + s(lat_{ja}, lon_{ja}) + f_{a}(\nu_{ja}) spei_{ja} + \epsilon_{ija} \label{eqn:GAM}
\end{equation}

Where $i$ indexes individuals, $j$ indexes DHS sites, and $a$ indexes agro-ecological zones. In this model, $y_{ija}$ is a given child's HAZ score, $\beta_0$ is a fixed intercept, $X_{ija}$ is a matrix of individual and household covariates, modified by a vector of coefficients $\beta$, $s(lat_{ja}, lon_{ja})$ is a spatially varying effect estimated by a spherical spline basis \cite{Wahba1982}, and $f_{a}()$ is a spline function that determines coefficient for the 24-month SPEI based on the amount of uncultivated land cover $\nu_{ja}$, estimated separately for each AEZ.  The basis we use for the varying coefficient function $f_{a}()$ is estimated using thin plate splines \cite{Duchon1977}, and the smoothing parameter for this term is estimated through Generalized Cross Validation (GCV) \cite{Wood2017}.

To more precisely estimate the effect of drought on child stunting, we control for a number of individual and household factors that affect stunting outcomes typically included in analyses of HAZ scores \cite{brown2020empirical}.  Specifically, we control for the child's age, the child's birth order, the size of the household the child lives in, the sex of the child, the mother's years of education, the household's toilet facility, the interview year, the age of the household head, the sex of the household head, the month of the child's birth, which can be a source of measurement error in estimating the child's HAZ score \cite{larsen2019misreporting}, as well as the household wealth index, normalized to be comparable across surveys \cite{Rutstein2014c}.

\subsection{Mapping Where to Target Conservation Interventions}
While our observations from the DHS and our model measure malnutrition in terms of HAZ scores, HAZ scores alone are not sufficient to estimate where uncultivated land cover is most important for drought resilience.  Both current rates of malnutrition as well as current population distributions are crucial for estimating the human benefit provided by local ecosystem services and are not captured by HAZ scores.  Thus, in AEZs where uncultivated land was associated with drought resilience, we estimate for each pixel the number of additional children that would be stunted during a drought in the absence of uncultivated land cover, using the following equation:

\begin{equation}
  pop * (stunting(HAZ_{0}) - stunting(HAZ_{\nu})) \label{eqn:stunting}
\end{equation}

Where $pop$ is the current under-5 population count in a given pixel, $stunting()$ is an equation to estimate rates of stunting from HAZ scores, $HAZ_{0}$ is the mean HAZ score in a pixel under drought conditions with no uncultivated land cover, and $HAZ_{\nu}$ is the mean HAZ score in a pixel with current rates of uncultivated land cover.

To estimate both $HAZ_{0}$ and $HAZ_{\nu}$, we first need estimates of prevailing mean HAZ scores across the continent, or $HAZ_{current}$.  We derive these from a recent analysis of rates of stunting in Africa \cite{Osgood-Zimmerman2018}.  Because this analysis estimated rates of stunting for the years 2000-2015, we the annualized rate of change (AROC) trend extrapolation method common in epidemiology to conduct a forecast to the year 2020 \cite{Fullman2017, Osgood-Zimmerman2018}.  We then convert the estimated rates of stunting to HAZ scores using the quantile function of the normal distribution.  In our calculations based on the normal distribution, we use the observed standard deviation in HAZ scores for our dataset ($\sigma = 1.62$).  This is because, overall standard deviations in HAZ scores have been observed to vary independently of mean HAZ scores and to not change significantly over time \cite{Mei2007}.  Furthermore, our estimated value matches previous literature on the standard deviation of HAZ scores in surveys in Africa \cite{Mei2007}.

Having derived current mean HAZ scores across the continent, we use our regression results in equation \ref{eqn:GAM} to estimate the marginal effect on mean HAZ scores of a drought with an SPEI of -2.5, both with prevailing rates of uncultivated land and in the absence of uncultivated land.  We estimate $HAZ_{\nu}$ as current HAZ scores plus the estimated decrease under drought, or $HAZ_{current} + f_{a}(\nu) spei$ where $f_{a}$ is the AEZ-specific varying-coefficient function, $\nu$ is equal to the prevailing rate of uncultivated land, and $spei = -2.5$.  We estimated $HAZ_{0}$ similarly, except where $\nu$ is equal to 0, indicating no uncultivated land.

We then convert these HAZ estimates to rates of stunting using the cumulative density function of the normal distribution, estimating the fraction of the distribution less than -2, given the HAZ scores as $\mu$ and the empirically derived $\sigma$ of 1.62.  This fraction of the distribution less than -2 is the rate of stunting, and $stunting(HAZ_{0}) - stunting(HAZ_{\nu})$ is the potential increase in rates of stunting without uncultivated land cover.

Finally, we then multiply this increase in rates of stunting by the number of under-5 children in each pixel, with age-specific population data derived from WorldPop \cite{Tatem2017}.  This yields the spatial distribution of children modeled to be stunting during a drought in the absence of ecosystem services from uncultivated land cover.  We then aggregate this count per pixel by square kilometer and by country.

Then, in order to highlight areas where environmental conservation can achieve multiple climate-relevant conservation goals, we compare our predictions of where conservation interventions can contribute to climate-resilient nutrition with a map of conservation priorities to preserve biodiversity under climate change \cite{hannah2020}.  This comparison map of biodiversity priorities was the result of modeling the expected habitat ranges of vertebrate and plant species in the period 2060-2080 under the Representative Concentration Pathway (RCP) 8.5 \cite{hannah2020}.

\section{Results}
\subsection{Covariate Balancing}
After estimating weights using CBGPS, the correlation between uncultivated land cover and the various confounding variables that we attempted to control for was significantly reduced.  Table \ref{tab:CBPSsum} shows the reduction in correlation between these variables based on the weighting.  Before weighting, many of these covariates where highly correlated with uncultivated land cover, with absolute values as high as 0.7 for some variables.  Population density in particular was highly correlated with land cover across nearly all AEZs.  After weighting, many of these correlations were reduced, with most variables having a correlation of less than 0.05 with uncultivated land cover across all AEZs.

\begin{table}[h!]
	\begin{center}
		\begin{tabular}{r | c c c c c c c c c c c c c c c c }
&	\multicolumn{2}{c}{Import Value} &	\multicolumn{2}{c}{Population} &	\multicolumn{2}{c}{Subnational}	&	\multicolumn{2}{c}{Time to Travel}	\\
&	\multicolumn{2}{c}{Per Capita} &	\multicolumn{2}{c}{Density} &	\multicolumn{2}{c}{GDP Per Capita}	&	\multicolumn{2}{c}{to Major City}	\\
	\cmidrule(lr){2-3}\cmidrule(lr){4-5} \cmidrule(lr){6-7}\cmidrule(lr){8-9}
      AEZ	&	\textit{Unwgtd.}	&	\textit{Wgtd.}	&	\textit{Unwgtd.}	&	\textit{Wgtd.}	&	\textit{Unwgtd.}	&	\textit{Wgtd.}	&	\textit{Unwgtd.}	&	\textit{Wgtd.}	\\
\hline									
Arid	&	0.22	&	0.01	&	0.27	&	0 &	0.16	&	-0.01	&	-0.15	&	0	\\
Tropical Forest	&	0.1	&	0.1	&	-0.47	&	0.04&	0.19	&	0.08	&	0.31	&	-0.03	\\
Montane	&	0.37	&	0.03	&	-0.64	&	-0.14&	0.17	&	-0.04	&	0.33	&	0.12	\\
Northern Savanna	&	0.02	&	0.03	&	-0.45	&	0.02&	-0.16	&	0.02	&	0.24	&	0.01	\\
Northern Woodland	&	0.16	&	-0.03	&	-0.41	&	-0.01&	-0.03	&	-0.04	&	0.12	&	0.03	\\
Southern Savanna	&	0.45	&	0.01	&	-0.62	&	-0.03&	0.47	&	0.02	&	0.24	&	0.05	\\
Southern Woodland	&	0.46	&	-0.09	&	-0.7	&	0.18&	0.22	&	0.06	&	0.17	&	0.05	\\	
		\end{tabular}
    \caption{Summary of correlation between uncultivated land cover and confounding variables with no weighting (\textit{Unwgtd.}) and after weighting using CBGPS (\textit{Wgtd.}).}
	\label{tab:CBPSsum}
	\end{center}
\end{table}

\subsection{Role of Uncultivated Land Cover in Moderating Drought Vulnerability by AEZ}

Having estimated the model, our main parameters of interest are the varying coefficients for how uncultivated land cover affects the impact of drought in each AEZ.  Thus, we graph those effects here in Figure \ref{fig:naturaleffect}, and include full model results in the Appendix.

\begin{figure}[h!]
	\begin{center}
	\includegraphics[width=\linewidth]{AEZ_effects.png}
	\end{center}
  \caption{Effect of droughts on child nutrition outcomes by agro-ecological zone (AEZ), varying as a function of the percent of nearby land cover that is uncultivated.  In arid, savanna, and montane zones, more uncultivated land is associated with greater drought vulnerability, while in woodland zones, more uncultivated land is associated with less drought vulnerability.  Error bands indicate the 95\% confidence interval. Colors correspond to AEZs (See Figure \ref{fig:AEZmap})}
	\label{fig:naturaleffect}
\end{figure}

Figure \ref{fig:naturaleffect} shows how the coefficient for the 24-month SPEI varies as a function of the percent of nearby uncultivated land.  The error band around the parameter indicates the 95\% confidence interval.  Thus, areas where the error band does not cross 0 (at the dotted line) indicates that, at that level of uncultivated land cover, precipitation anomalies have a statistically significant effect on child nutrition outcomes.

In many AEZs, the functions for the varying-coefficients slope upwards, indicating that increasing rates of uncultivated land cover are associated with a larger coefficient and thus greater drought vulnerability. For example, in arid AEZs with 0\% of nearby land uncultivated, the coefficient for the effect of SPEI is 0, indicating that droughts have little effect on local HAZ scores.  However, in the same AEZ with 100\% of nearby land uncultivated, the coefficient for the effect of SPEI is 0.2, meaning that an SPEI of -2 is associated with a commensurate decrease in HAZ scores of -0.4.

In the woodland AEZs of both northern and southern Africa, increasing rates of uncultivated land cover are associated with a smaller coefficient and thus less drought vulnerability.  At low levels of uncultivated land cover in both northern and southern sub-forest Africa, a moderate drought (SPEI = -2) decreases mean HAZ scores by 0.2 to 0.4, whereas at high levels of uncultivated land cover, a similar drought has no significant effect on nutrition outcomes.

\subsection{Modeling Ecosystem Service Dependence Over Space}
Our model indicates that in semi-humid woodland parts of Africa, uncultivated land buffers child nutrition from the effects of drought.  Thus, focusing on these AEZs, we contextualize the model by estimating the increase in the number of under-5 children that would become stunted in the absence of uncultivated land during a drought.

\begin{figure}[h!]
	\begin{center}
		\includegraphics[width=\linewidth]{AfricaEffect.png}
    \caption{\textbf{A}) The number of additional children per square kilometer in woodland AEZs who would become stunted during a drought in the absence of uncultivated land cover.  \textbf{B}) The same figure, aggregated by country rather than calculated per square kilometer.  Countries without woodland areas are shown in gray.}
		\label{fig:AfricaEffect}
	\end{center}
\end{figure}

Figure \ref{fig:AfricaEffect} shows the number of additional children that would become stunted during a drought in the absence of uncultivated land cover, based on current land cover conditions and rates of stunting, estimated per square kilometer and aggregated to the country level.  The areas that would see an increase in stunting in the absence of local uncultivated land were mostly the woodlands of Africa, such as the Guinean forest-savanna mosaic of Northern and Western Africa as well as the Miombo woodlands of Southern Africa. Examining the potential increase in stunted child under drought in each of these AEZs shows that many of them would be located in the woodlands of southern Democratic Republic of the Congo (DRC), central Nigeria as well as in parts of Mozambique, Malawi, and southern Tanzania.  Throughout Africa, an additional 1.5 million children would be stunted under drought without local ecosystem services.  The countries that currently see the most benefit to child nutrition from local ecosystem services are the DRC, Mozambique, Nigeria, and Tanzania.

\subsection{Comparison With Biodiversity Conservation Priorities}

\begin{figure}[h!]
	\begin{center}
		\includegraphics[width=0.8\linewidth]{Bivariate_Map.png}
		\caption{Map of the intersection of areas that are a priority for conservation under climate change for two goals of preserving biodiversity and ensuring resilient food security in woodland AEZs of Africa.  Areas in blue are in the top tercile of priority for biodiversity conservation, because these are areas most likely to be critical habitat for vertebrates and plants under future climate change.  Areas in orange are in the top tercile of priority for food security, because these are areas where uncultivated land provides ecosystem services that prevent drought-related child stunting.  Areas in black are in the top tercile of priority for both food security and biodiversity.}
		\label{fig:Bivariate_Map}
	\end{center}
\end{figure}

Examining the overlap between two conservation goals under climate change highlights landscapes throughout Africa where conservation interventions could meet both biodiversity and food security goals (See Figure \ref{fig:Bivariate_Map}).  These included areas in Benin, northern Uganda and southern South Sudan, the Katanga region of the DRC, the mouth of the Congo River, the coastal area the Mozambique-Tanzania border, Eswatini and nearby parts of Mozambique and South Africa, as well as parts of Madagascar. There are also many landscapes throughout the continent that are priorities for one of the two goals, but not both.

Comparing the areas in the top tercile of food security priority with the current distribution of protected areas (See Figure \ref{fig:Bivariate_Map_WDPA} in the Supplement) \cite{UNEP-WCMC2021}, only 10.96\% of these areas are currently protected.  Of those areas, 23.34\% are in national parks, and 17.79\% are in areas that permit the sustainable use of natural resources.  While the analysis of biodiversity conservation priorities was focused on areas that are not currently protected, many of these areas are located near to current protected areas.

\section{Discussion}

This paper assessed how the prevalence of uncultivated land cover moderates the impact of drought on child nutrition outcomes throughout several agro-ecological zones in Africa.  We took care to control for the potential confounding effects of several factors that could influence both the presence of uncultivated land as well as drought vulnerability.  We found that the manner in which uncultivated land cover moderated the effect of drought on child nutrition outcomes varied by AEZ, and that there is an observable safety net effect in semi-humid woodland landscapes throughout the continent, although uncultivated land cover is associated with greater drought vulnerability in arid and savanna AEZs.  Finally, examining the potential impact of droughts without uncultivated land and the ecosystem services it provides shows that millions of children are dependent on ecosystem services to meet their nutrition needs in times of drought.

A major contribution of this paper to the literature is its scale.  Most other studies of the role in ecosystem services in buffering human well-being from climate shocks tends to focus on case studies \cite{Debela2012} as well as use hypothetical scenarios \cite{Robledo2012} or retrospective analyses \cite{Muller2008}.  This paper provides a large scale analysis of observed nutrition outcomes during varying levels of drought as well as across sites with varying access to ecosystem services.  Performing an analysis at this scale allowed us to compare how uncultivated land affects drought vulnerability across many agro-ecological zones and aid in conservation priority setting across Africa.

An important aspect of this analysis was using weighting to ameliorate the effects of potential confounding variables.  Because we controlled for the effects of several demographic and economic variables, we can more confidently ascribe the observed drought mitigation to the land cover itself rather than to another factor that is correlated with land cover.  However, given that weighting each covariate to achieve a correlation of perfectly 0 would be either impossible or would require extreme weights, we did not reduce the correlation between our confounding variables and natural land cover all the way to 0 (See Table \ref{tab:CBPSsum}).  Nevertheless, we diminished the correlation to the extent that a causal interpretation of the observed mitigation effect of natural land cover is now more plausible. Moreover, we validated the robustness of our weighting by censoring the weights at the 80th and 90th percentile and getting similar results, confirming that the observed effects were not due to extreme weights on a small number of observations.

While the model estimated the moderating effect of natural land cover on drought vulnerability as varying across AEZs, we found that uncultivated land cover played a similar function in ecologically similar zones.  In both northern-hemisphere and southern-hemisphere savanna zones, greater uncultivated land cover was associated with greater drought vulnerability.  On the other hand, in the ecologically similar but geographically disjointed woodland zones, natural land cover had a safety net effect during drought.  The fact that ecologically similar AEZs were modeled as having similar effects in terms of drought vulnerability, even though they were modeled with independently estimated smoothing splines, suggests that this effect is real and is ecologically based.

We found that in arid and savanna AEZs, a greater rate of uncultivated land cover was in fact associated with greater drought vulnerability.  This could be due to the fact that much of the vegetation in these areas is annual grasses, which, like annual crops, are highly affected by droughts because they grow entirely within one season and do not have deep taproots like woody vegetation in more humid areas.  Moreover, arid and savanna landscapes provide less wild foods or other provisioning services compared to other vegetation regimes, and so are primarily used for grazing livestock.  Similarly, many regulating and supporting ecosystem services provided by natural land cover, such as wind breaking, shading and temperature regulation, and moisture retention are specifically a function of trees \cite{Reed2016}.  Thus, areas lacking in trees may not be able to provide the safety net effect that more forested areas have.  For very humid and mesic areas with closed-canopy tropical forests, other the other hand, drought does not have a significant effect on stunting at any level of uncultivated land cover.  Our results suggest that, in this AEZ, nutrition is unaffected even if precipitation is well below historic norms and, if anything, increased stunting may be caused by excess rainfall in certain landscapes.

In contrast to both savannas and tropical forests, in the open-canopy woodlands on both northern and southern Africa uncultivated land is associated with decreased drought vulnerability.  This may be because these areas present a middle ground, where rainfall levels are low enough that a drought can affect food production and lead to increases in stunting, but rainfall is still high enough that in uncultivated areas there is both the biodiversity and biomass to provide a safety net.  Moreover, these mixed woodland landscapes between open grasslands and dense forests can support a wide variety of land cover types, and farmers frequently shape the landscape to include a variety of vegetation communities and maximize a diversity of food sources \cite{fairhead1996misreading}.  While we have found that these uncultivated areas are generally associated with decreased drought vulnerability in woodland areas, there is likely significant local heterogeneity in the exact role they play in local livelihoods, with some areas being more actively managed and others being more abandoned to problems like degradation and bush encroachment \cite{o2014bush}.  Thus, the specific benefits of uncultivated land are likely highly dependent on how local people utilize, manage, and interact with the landscape.

While the association between natural land cover and reduced drought vulnerability in woodland AEZs is certainly suggestive that people are relying on ecosystem services as a safety net, this analysis cannot speak directly to the particular pathways through which people are benefiting from uncultivated land.  Nevertheless, several lines of evidence suggest that wild foods are an important component.  Previous work across multiple African countries has found that greater natural land cover is associated with greater collection of wild foods \cite{Cooper2018a}.  Moreover, while a comprehensive analysis of where people collect wild foods has yet to be conducted across the continent, examples of wild foods playing an important role in peoples diets in woodland parts of Africa are abundant.  The woodland areas of west Africa closely match the distribution of the widely consumed Shea tree (\textit{Vitellaria paradoxa}) \cite{Naughton2015, Naughton2017}, the woodlands of northern Uganda have been found to have unusually high rates of wild food consumption \cite{cooper2017vs}, the eastern Usambara mountains of Tanzania have at least 92 wild foods species consumed by local people \cite{powell2013wild}, and there are examples of literature documenting wild food consumption in woodland parts of South Africa \cite{garekae2020foraging}, DRC \cite{de2004value}, Zimbabwe \cite{zinyama1990use}, and Burkina Faso \cite{lamien2008importance}.  Countering these examples, one of the only other multinational analyses of the role of provisioning ecosystem services as a buffer during shocks found that households did not rank forest resources as a very important resource during shocks \cite{Wunder2014}. However, this study did not focus on woodland areas in particular.  Moreover, it may be that people are not shifting their consumption to wild foods during shocks, but rather that livelihoods that are more dependent on wild foods are simply less affected by climatological shocks like drought.

Combining prevailing land cover conditions, population density, and rates of child stunting, we identified the areas where uncultivated is most critical for drought resilience, and found hot spots in woodland areas across the continent (See Figure \ref{fig:AfricaEffect}).  Many of the areas identified, from Nigeria, to the DRC to Mozambique are places frequently identified by the Famine Early Warning Systems Network (FEWSNET) as being in conditions of poor food security \cite{FEWSNET2017, FEWSNET2018, FEWSNET2020}.  Moreover, some of these areas, such as northern Mozambique, are less ecologically conducive to cattle raising, depriving people of a common safety net in more arid or grassland rural areas \cite{mabiso2014food}.

Finally, we used our model to map where conservation interventions could have the largest impact on reducing child malnutrition under an increasingly drought-prone climate, and compared this map with the results of a recent study examining conservation priorities for conserving plant and vertebrate diversity under climate change \cite{hannah2020}.  The resulting map (See Figure \ref{fig:Bivariate_Map}) highlights many landscapes where conservation could synergistically help meet SDGs 2 and 15 - to improve food security and preserve biodiversity.  Aside from being in woodland AEZs, these landscapes tend to be mildly populated areas, often near uninhabited existing national parks and protected areas, such as Pendjari National Park in Benin, Murchison Falls National Park in Uganda, or Kruger National Park in South Africa and Parque Nacional de Limpopo in Mozambique.  In these areas, people-centered conservation schemes such as community based forest management could support better nutrition and biodiversity outcomes under a changing climate.

\section{Conclusion}
These findings are have important implications for the study of food security, climate change vulnerability, and environmental conservation.  We showed that uncultivated land can be a critical part of reducing climate change vulnerability, but the specific role that nature plays is highly context-specific.  While mapping ecosystem services has traditionally focused on variables like carbon stocks and biodiversity hotspots, this analysis shows that the contributions of ecosystem services to food security can also be mapped to support improved nutrition.  Given the increasing threat of a more drought prone world under climate change \cite{Dai2013} combined with the severe precarity of Africa's agrarian poor, dampening the effects of drought and providing alternative food and income sources when agriculture fails may indeed be one of nature's most important contributions to people.

\bibliographystyle{apalike}
\bibliography{library}

\setcounter{section}{0}
\renewcommand{\thetable}{A\arabic{section}}
\section*{Appendix} \label{AppendixA}
\setcounter{table}{0}
\setcounter{figure}{0}
\renewcommand{\thetable}{A\arabic{table}}
\renewcommand{\thefigure}{A\arabic{figure}}

\section{Full Model Results}

\begin{longtable}{l c }
\hline
 &  \\
\hline
\endfirsthead
\hline
 &  \\
\hline
\endhead
\hline
\endfoot
\hline
\multicolumn{2}{l}{\scriptsize{$^{***}p<0.001$, $^{**}p<0.01$, $^*p<0.05$}}\\
\caption{Statistical models}
\label{table:coefficients}
\endlastfoot
age                              & $-0.02^{***}$  \\
                                 & $(0.00)$       \\
birth\_order                     & $0.01^{***}$   \\
                                 & $(0.00)$       \\
hhsize                           & $-0.00$        \\
                                 & $(0.00)$       \\
sexFemale                        & $-17.07^{***}$ \\
                                 & $(1.44)$       \\
sexMale                          & $-17.19^{***}$ \\
                                 & $(1.44)$       \\
mother\_years\_ed                & $0.03^{***}$   \\
                                 & $(0.00)$       \\
toiletNo Facility                & $-0.16^{***}$  \\
                                 & $(0.01)$       \\
toiletOther                      & $-0.14^{***}$  \\
                                 & $(0.03)$       \\
toiletPit Latrine                & $-0.13^{***}$  \\
                                 & $(0.01)$       \\
interview\_year                  & $0.01^{***}$   \\
                                 & $(0.00)$       \\
as.factor(calc\_birthmonth)2     & $-0.02$        \\
                                 & $(0.02)$       \\
as.factor(calc\_birthmonth)3     & $0.04^{*}$     \\
                                 & $(0.02)$       \\
as.factor(calc\_birthmonth)4     & $0.03^{*}$     \\
                                 & $(0.02)$       \\
as.factor(calc\_birthmonth)5     & $0.03^{*}$     \\
                                 & $(0.02)$       \\
as.factor(calc\_birthmonth)6     & $0.15^{***}$   \\
                                 & $(0.02)$       \\
as.factor(calc\_birthmonth)7     & $0.11^{***}$   \\
                                 & $(0.02)$       \\
as.factor(calc\_birthmonth)8     & $0.18^{***}$   \\
                                 & $(0.02)$       \\
as.factor(calc\_birthmonth)9     & $0.17^{***}$   \\
                                 & $(0.02)$       \\
as.factor(calc\_birthmonth)10    & $0.23^{***}$   \\
                                 & $(0.02)$       \\
as.factor(calc\_birthmonth)11    & $0.23^{***}$   \\
                                 & $(0.02)$       \\
as.factor(calc\_birthmonth)12    & $0.44^{***}$   \\
                                 & $(0.02)$       \\
head\_age                        & $0.00^{***}$   \\
                                 & $(0.00)$       \\
head\_sexMale                    & $-0.07^{***}$  \\
                                 & $(0.01)$       \\
wealth\_norm                     & $0.54^{***}$   \\
                                 & $(0.02)$       \\
AEZ\_newafr.forest.4             & $-0.11^{***}$  \\
                                 & $(0.03)$       \\
AEZ\_newafr.high.7               & $-0.22^{***}$  \\
                                 & $(0.03)$       \\
AEZ\_newnafr.sav.5               & $0.00$         \\
                                 & $(0.02)$       \\
AEZ\_newnafr.subforest.8         & $0.03$         \\
                                 & $(0.03)$       \\
AEZ\_newsafr.subforest.9         & $0.06^{*}$     \\
                                 & $(0.03)$       \\
AEZ\_newseafr.sav.6              & $-0.17^{***}$  \\
                                 & $(0.03)$       \\
EDF: s(latitude,longitude)       & $45.17^{***}$  \\
                                 & $(49.00)$      \\
EDF: s(natural):afr.arid.123     & $3.24^{***}$   \\
                                 & $(3.74)$       \\
EDF: s(natural):afr.forest.4     & $3.20^{**}$    \\
                                 & $(3.74)$       \\
EDF: s(natural):nafr.sav.5       & $2.73^{***}$   \\
                                 & $(3.16)$       \\
EDF: s(natural):seafr.sav.6      & $3.20^{***}$   \\
                                 & $(3.75)$       \\
EDF: s(natural):afr.high.7       & $2.76^{***}$   \\
                                 & $(3.20)$       \\
EDF: s(natural):nafr.subforest.8 & $2.00^{***}$   \\
                                 & $(2.00)$       \\
EDF: s(natural):safr.subforest.9 & $2.97^{***}$   \\
                                 & $(3.46)$       \\
\hline
AIC                              & 890428.85      \\
BIC                              & 891421.48      \\
Log Likelihood                   & -445118.15     \\
Deviance                         & 16.37          \\
Deviance explained               & 0.48           \\
Dispersion                       & 0.00           \\
R$^2$                            & 0.11           \\
GCV score                        & 0.00           \\
Num. obs.                        & 221885         \\
Num. smooth terms                & 8              \\
\end{longtable}


\newpage

\section{Model Results With Weights Censored at the 90th Percentile}
\begin{figure}[h!]
	\begin{center}
	\includegraphics[width=\linewidth]{AEZ_effects_q90.png}
	\end{center}
	\caption{Effect of droughts on child nutrition outcomes by agro-ecological zone (AEZ), varying as a function of the percent of nearby land cover that is uncultivated, estimated with weights censored at the 90th percentile.  Error bands indicate the 95\% confidence interval.}
\end{figure}


\begin{longtable}{l c}
\hline
 & Model 1 \\
\hline
\endfirsthead
\hline
 & Model 1 \\
\hline
\endhead
\hline
\endfoot
\hline
\multicolumn{2}{l}{\scriptsize{$^{***}p<0.001$; $^{**}p<0.01$; $^{*}p<0.05$}}\\
\caption{Parameter estimates for Generalized Additive Model estimating the varying coefficient of SPEI, with CBGPS weights censored at the 90th percentile.}
\label{table:coefficients}
\endlastfoot \\
age                              & $-0.02^{***}$  \\
                                 & $(0.00)$       \\
birth\_order                     & $0.01^{***}$   \\
                                 & $(0.00)$       \\
hhsize                           & $-0.00$        \\
                                 & $(0.00)$       \\
sexFemale                        & $-20.63^{***}$ \\
                                 & $(1.42)$       \\
sexMale                          & $-20.76^{***}$ \\
                                 & $(1.42)$       \\
mother\_years\_ed                & $0.03^{***}$   \\
                                 & $(0.00)$       \\
toiletNo Facility                & $-0.14^{***}$  \\
                                 & $(0.01)$       \\
toiletOther                      & $-0.12^{***}$  \\
                                 & $(0.03)$       \\
toiletPit Latrine                & $-0.11^{***}$  \\
                                 & $(0.01)$       \\
interview\_year                  & $0.01^{***}$   \\
                                 & $(0.00)$       \\
as.factor(calc\_birthmonth)2     & $-0.02$        \\
                                 & $(0.02)$       \\
as.factor(calc\_birthmonth)3     & $0.03^{*}$     \\
                                 & $(0.02)$       \\
as.factor(calc\_birthmonth)4     & $0.03^{*}$     \\
                                 & $(0.02)$       \\
as.factor(calc\_birthmonth)5     & $0.04^{**}$    \\
                                 & $(0.02)$       \\
as.factor(calc\_birthmonth)6     & $0.12^{***}$   \\
                                 & $(0.02)$       \\
as.factor(calc\_birthmonth)7     & $0.09^{***}$   \\
                                 & $(0.02)$       \\
as.factor(calc\_birthmonth)8     & $0.15^{***}$   \\
                                 & $(0.02)$       \\
as.factor(calc\_birthmonth)9     & $0.15^{***}$   \\
                                 & $(0.02)$       \\
as.factor(calc\_birthmonth)10    & $0.21^{***}$   \\
                                 & $(0.02)$       \\
as.factor(calc\_birthmonth)11    & $0.21^{***}$   \\
                                 & $(0.02)$       \\
as.factor(calc\_birthmonth)12    & $0.35^{***}$   \\
                                 & $(0.02)$       \\
head\_age                        & $0.00^{***}$   \\
                                 & $(0.00)$       \\
head\_sexMale                    & $-0.03^{***}$  \\
                                 & $(0.01)$       \\
wealth\_norm                     & $0.50^{***}$   \\
                                 & $(0.02)$       \\
AEZ\_newafr.forest.4             & $-0.09^{**}$   \\
                                 & $(0.03)$       \\
AEZ\_newafr.high.7               & $-0.20^{***}$  \\
                                 & $(0.03)$       \\
AEZ\_newnafr.sav.5               & $0.01$         \\
                                 & $(0.02)$       \\
AEZ\_newnafr.subforest.8         & $0.05$         \\
                                 & $(0.03)$       \\
AEZ\_newsafr.subforest.9         & $0.04$         \\
                                 & $(0.03)$       \\
AEZ\_newseafr.sav.6              & $-0.14^{***}$  \\
                                 & $(0.03)$       \\
EDF: s(latitude,longitude)       & $48.16^{***}$  \\
                                 & $(49.00)$      \\
EDF: s(natural):afr.arid.123     & $3.26^{***}$   \\
                                 & $(3.77)$       \\
EDF: s(natural):afr.forest.4     & $3.32^{*}$     \\
                                 & $(3.88)$       \\
EDF: s(natural):nafr.sav.5       & $3.33^{***}$   \\
                                 & $(3.90)$       \\
EDF: s(natural):seafr.sav.6      & $3.39^{***}$   \\
                                 & $(3.97)$       \\
EDF: s(natural):afr.high.7       & $2.46^{**}$    \\
                                 & $(2.79)$       \\
EDF: s(natural):nafr.subforest.8 & $2.00^{***}$   \\
                                 & $(2.00)$       \\
EDF: s(natural):safr.subforest.9 & $3.32^{*}$     \\
                                 & $(3.89)$       \\
\hline
AIC                              & $833514.31$    \\
BIC                              & $834547.96$    \\
Log Likelihood                   & $-416656.90$   \\
Deviance                         & $31.07$        \\
Deviance explained               & $0.49$         \\
Dispersion                       & $0.00$         \\
R$^2$                            & $0.11$         \\
GCV score                        & $0.00$         \\
Num. obs.                        & $221885$       \\
Num. smooth terms                & $8$            \\
\end{longtable}


\newpage

\section{Model Results With Weights Censored at the 80th Percentile}
\begin{figure}[h!]
	\begin{center}
	\includegraphics[width=\linewidth]{AEZ_effects_q80.png}
	\end{center}
	\caption{Effect of droughts on child nutrition outcomes by agro-ecological zone (AEZ), varying as a function of the percent of nearby land cover that is uncultivated, estimated with weights censored at the 90th percentile.  Error bands indicate the 95\% confidence interval.}
\end{figure}


\begin{longtable}{l c}
\hline
 & Model 1 \\
\hline
\endfirsthead
\hline
 & Model 1 \\
\hline
\endhead
\hline
\endfoot
\hline
\multicolumn{2}{l}{\scriptsize{$^{***}p<0.001$; $^{**}p<0.01$; $^{*}p<0.05$}}\\
\caption{Parameter estimates for Generalized Additive Model estimating the varying coefficient of SPEI, with CBGPS weights censored at the 80th percentile.}
\label{table:coefficients}
\endlastfoot \\
age                              & $-0.02^{***}$  \\
                                 & $(0.00)$       \\
birth\_order                     & $0.01^{***}$   \\
                                 & $(0.00)$       \\
hhsize                           & $-0.00$        \\
                                 & $(0.00)$       \\
sexFemale                        & $-19.63^{***}$ \\
                                 & $(1.43)$       \\
sexMale                          & $-19.76^{***}$ \\
                                 & $(1.43)$       \\
mother\_years\_ed                & $0.03^{***}$   \\
                                 & $(0.00)$       \\
toiletNo Facility                & $-0.15^{***}$  \\
                                 & $(0.01)$       \\
toiletOther                      & $-0.13^{***}$  \\
                                 & $(0.03)$       \\
toiletPit Latrine                & $-0.12^{***}$  \\
                                 & $(0.01)$       \\
interview\_year                  & $0.01^{***}$   \\
                                 & $(0.00)$       \\
as.factor(calc\_birthmonth)2     & $-0.02$        \\
                                 & $(0.02)$       \\
as.factor(calc\_birthmonth)3     & $0.03^{*}$     \\
                                 & $(0.02)$       \\
as.factor(calc\_birthmonth)4     & $0.03^{*}$     \\
                                 & $(0.02)$       \\
as.factor(calc\_birthmonth)5     & $0.04^{*}$     \\
                                 & $(0.02)$       \\
as.factor(calc\_birthmonth)6     & $0.13^{***}$   \\
                                 & $(0.02)$       \\
as.factor(calc\_birthmonth)7     & $0.10^{***}$   \\
                                 & $(0.02)$       \\
as.factor(calc\_birthmonth)8     & $0.15^{***}$   \\
                                 & $(0.02)$       \\
as.factor(calc\_birthmonth)9     & $0.15^{***}$   \\
                                 & $(0.02)$       \\
as.factor(calc\_birthmonth)10    & $0.22^{***}$   \\
                                 & $(0.02)$       \\
as.factor(calc\_birthmonth)11    & $0.22^{***}$   \\
                                 & $(0.02)$       \\
as.factor(calc\_birthmonth)12    & $0.38^{***}$   \\
                                 & $(0.02)$       \\
head\_age                        & $0.00^{***}$   \\
                                 & $(0.00)$       \\
head\_sexMale                    & $-0.05^{***}$  \\
                                 & $(0.01)$       \\
wealth\_norm                     & $0.51^{***}$   \\
                                 & $(0.02)$       \\
AEZ\_newafr.forest.4             & $-0.10^{**}$   \\
                                 & $(0.03)$       \\
AEZ\_newafr.high.7               & $-0.20^{***}$  \\
                                 & $(0.03)$       \\
AEZ\_newnafr.sav.5               & $0.01$         \\
                                 & $(0.02)$       \\
AEZ\_newnafr.subforest.8         & $0.05$         \\
                                 & $(0.03)$       \\
AEZ\_newsafr.subforest.9         & $0.05$         \\
                                 & $(0.03)$       \\
AEZ\_newseafr.sav.6              & $-0.15^{***}$  \\
                                 & $(0.03)$       \\
EDF: s(latitude,longitude)       & $47.94^{***}$  \\
                                 & $(49.00)$      \\
EDF: s(natural):afr.arid.123     & $3.24^{***}$   \\
                                 & $(3.75)$       \\
EDF: s(natural):afr.forest.4     & $3.24^{*}$     \\
                                 & $(3.78)$       \\
EDF: s(natural):nafr.sav.5       & $3.16^{***}$   \\
                                 & $(3.68)$       \\
EDF: s(natural):seafr.sav.6      & $3.29^{***}$   \\
                                 & $(3.84)$       \\
EDF: s(natural):afr.high.7       & $2.36^{**}$    \\
                                 & $(2.64)$       \\
EDF: s(natural):nafr.subforest.8 & $2.00^{***}$   \\
                                 & $(2.00)$       \\
EDF: s(natural):safr.subforest.9 & $3.16^{***}$   \\
                                 & $(3.69)$       \\
\hline
AIC                              & $842174.38$    \\
BIC                              & $843199.14$    \\
Log Likelihood                   & $-420987.79$   \\
Deviance                         & $24.27$        \\
Deviance explained               & $0.49$         \\
Dispersion                       & $0.00$         \\
R$^2$                            & $0.11$         \\
GCV score                        & $0.00$         \\
Num. obs.                        & $221885$       \\
Num. smooth terms                & $8$            \\
\end{longtable}


\newpage

\section{Intersection of biodiversity and food security conservation priorities, with protected areas.}

\begin{figure}[h!]
	\begin{center}
		\includegraphics[width=\linewidth]{Bivariate_Map_WDPA.png}
		\caption{Map of the intersection of areas that are a priority for conservation under climate change for two goals of preserving biodiversity and ensuring resilient food security in woodland AEZs of Africa, with protected area boundaries shown.  Areas in blue are in the top tercile of priority for biodiversity conservation, because these are areas most likely to be critical habitat for vertebrates and plants under future climate change.  Areas in orange are in the top tercile of priority for food security, because these are areas where uncultivated land provides ecosystem services that prevent drought-related child stunting.  Areas in black are in the top tercile of priority for both food security and biodiversity.}
		\label{fig:Bivariate_Map_WDPA}
	\end{center}
\end{figure}


\end{document}
